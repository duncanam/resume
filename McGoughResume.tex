%%%%%%%%%%%%%%%%%%%%%%%%%%%%%%%%%%%%%%%%%
% Medium Length Professional CV
% LaTeX Template
% Version 2.0 (8/5/13)
%
% This template has been downloaded from:
% http://www.LaTeXTemplates.com
%
% Original author:
% Trey Hunner (http://www.treyhunner.com/)
%
% Important note:
% This template requires the resume.cls file to be in the same directory as the
% .tex file. The resume.cls file provides the resume style used for structuring the
% document.
%
%%%%%%%%%%%%%%%%%%%%%%%%%%%%%%%%%%%%%%%%%

%----------------------------------------------------------------------------------------
%	PACKAGES AND OTHER DOCUMENT CONFIGURATIONS
%----------------------------------------------------------------------------------------

\documentclass{resume} % Use the custom resume.cls style

\usepackage[left=0.3in,top=0.3in,right=0.4in,bottom=0.4in]{geometry} % Document margins
\usepackage{graphicx}

\name{Duncan McGough} % Your name
%\address{123 Broadway \\ City, State 12345} % Your address
%\address{123 Pleasant Lane \\ City, State 12345} % Your secondary addess (optional)
\address{(605)~$\cdot$~484~$\cdot$~3043 \\ duncan.mcgough@colorado.edu} % Your phone number and email
%\address{Space Grant}
%\address{\textbf{Full Resume and Work Experience}}  % full resume 


\begin{document}


%----------------------------------------------------------------------------------------
%	WORK EXPERIENCE SECTION
%----------------------------------------------------------------------------------------

%\begin{rSection}{Work Experience}
\begin{rSection}{Experience}

%---------------------------------------------------------------------------------------

	\begin{rSubsection}{Space Exploration Technologies Corporation (SpaceX)}{May 2018 - August 2018}{Satellite Development Intern}{Seattle, WA}
	\item {\small Produced and data-correlated a multiphysics ANSYS model of the hall-effect thruster used in the ``Tintin" and potential Starlink constellation. Model determined thermal expansion, temperatures, stress/strain, and vibration modes to inform the propulsion team on design choices for materials, sizing, and power as well as operation modes in orbit. Tested thruster coils in thermal vacuum chambers.}
	\item{\small Developed a Python script that utilizes a machine learning clustering algorithm to simplify phased-array antennas for input into Thermal Desktop. Reduced necessary modeling and computational time in Thermal Desktop as a result.}
	\item{\small Designed and constructed a test bed for determining performance of thermally-cycled heat pipes. Determined the initial performance of various heat pipe geometries and provided feedback for antenna thermal design. }
	\item{\small Produced a multiphysics thermal-structural ANSYS model of phased array antennas and their beamformer chips to provide design feedback for structural and antenna team on power, structure, and placement of components.}
	% you made a TD model of a tvac chamber with cold plates 
\end{rSubsection}

%---------------------------------------------------------------------------------------

\begin{rSubsection}{Roccor LLC }{May 2017 - August 2017}{Thermal Engineering Intern}{Longmont, CO}
\item {\small Engineered thermal vacuum chamber for deployable composite CubeSat radiators. Met NASA and Roccor project requirements and worked with machinists and welders to manufacture the chamber. Operated within project budgets and deadlines. Designed and simulated in SolidWorks and validated results by hand. Worked with FLIR instrumentation for chamber with IR optics. Developed Thermal Desktop model for system. Calibrated and tested chamber. Assisted thermal team with next-generation heat pipe development. Tested thermal solutions with high-vacuum systems.}
\end{rSubsection}

%---------------------------------------------------------------------------------------

\begin{rSubsection}{Aerospace Corporation: Senior Design Project}{August 2018 - Present}{Test and Integration Lead}{Boulder, CO}
\item {\small Modeled entire optical space object tracker in Solidworks and created assemblies and drawings for production. Selected CPU hardware and performed thermal analysis for CPU and power supply. Developed Python models for orbit tracking speed and required field of view. }
\end{rSubsection}

%---------------------------------------------------------------------------------------


\begin{rSubsection}{NASA Space Grant, 2017 Solar Eclipse Ballooning Project}{May 2016 - July 2016}{Colorado Space Grant Consortium Workshop Contractor}{Boulder, CO; Bozeman, MT}
\item {\small Provided individualized instruction to teams of university faculty and students with construction of high altitude solar live video, imaging payloads, and auto-tracking ground stations. Performed troubleshooting on radio still image and video transfer systems, Iridium Satellite communication payloads, and ground station payload tracking software.}
\end{rSubsection}


%---------------------------------------------------------------------------------------

%\begin{rSubsection}{Simply Mac Apple Specialist}{May 2015 - August 2015}{Sales Consultant}{Rapid City, SD}
%\item {\small Provide customers with knowledge about technical products, troubleshoot and diagnose electrical and hardware problems, sell technical hardware and software to customers, and work effectively in a team-driven atmosphere.} 
%\end{rSubsection}


%------------------------------------------------

%\end{rSection}


%----------------------------------------------------------------------------------------
%	Project Experience
%----------------------------------------------------------------------------------------

%\begin{rSection}{Project Experience}

%---------------------------------------------------------------------------------------

\begin{rSubsection}{COBRA ``Spaceshot" Suborbital Rocket Development Team}{March 2016 - January 2018}{Propulsion Lead}{Boulder, CO}
%\item {\small Manage propulsion subteam. Develop short and long-term goals. Interface with other subteams to insure product compatibility. Developing multistage rocket motor capable of 100km altitude.}
%\item {\small \textit{Previous Hybrid Rocket Revelopment Project:} Modeled oxidizer injector assembly and used CFD and FEA to validate designs. Calculated nozzle geometry and flow rates. Designed liquid oxidizer storage and delivery/injection systems for a hybrid rocket capable of 500N thrust.}
%\item {\small \textit{Previous General Team Responsibilities:} Assisted with the construction of a two-stage fiberglass rocket capable of reaching 10,000 feet AGL. Constructed body couplers, staging chassis, carbon fiber fins, payload and avionics bulkheads.}
\item {\small Manage propulsion subteam. Modeled oxidizer injector assembly and used CFD and FEA to validate designs. Calculated nozzle geometry and flow rates. Designed liquid oxidizer storage and delivery/injection systems for a hybrid rocket. } 
\end{rSubsection}


%---------------------------------------------------------------------------------------

%\begin{rSubsection}{CU COBRA Rocketry Team}{March 2016 - Present}{Team Member}{Boulder, CO}
%\item Assist with the construction of a two-stage fiberglass rocket capable of reaching 10,000 feet AGL. Constructed body couplers and staging chassis for subscale demonstration rocket. Formed and smoothed carbon fiber fins. Constructed payload and avionics bulkheads. 
%\end{rSubsection}

%---------------------------------------------------------------------------------------


\begin{rSubsection}{Gridded Electrostatic Ion Thruster Research BalloonSat Project}{January 2016 - May 2016}{Team Lead}{Boulder, CO}
\item {\small Led a team of CU Boulder students in an engineering projects class to design, test, and fly a high-altitude BalloonSat payload. Developed, researched, and constructed a low-cost gridded electrostatic ion thruster to test at altitude for deployment upon CubeSats. Designed and soldered power distribution for environmental sensor data collection and thruster. Analyzed data retrieved.}
\end{rSubsection}

%---------------------------------------------------------------------------------------
%\begin{rSubsection}{High-Altitude Balloon Projects}{December 2013 - May 2015}{Team Leads}{Rapid City, SD}
%\item {\small Led two teams of students to compete and win twice in successfully-funded design competitions for high-altitude ballooning projects that were launched and retrieved. Managed sub-teams. Designed and programmed live radio communication modules. Custom-built antenna and data transfer nodes and analyzed retrieved data. Programmed flight computer and all data collection instruments. Constructed payload module and assembled all technical devices for flight.}
%\end{rSubsection} 

%---------------------------------------------------------------------------------------

%\begin{rSubsection}{2015 High-Altitude Balloon Project}{December 2014 - May 2015}{Team Lead}{Rapid City, SD}
%\item Led a team of students to compete and win in a successfully-funded design competition for a high-altitude balloon project that was launched and retrieved. Managed the sub-teams. Designed and programmed the live radio communication modules. Custom-built the antenna and data transfer node and analyzed the retrieved data. 
%\end{rSubsection}

%---------------------------------------------------------------------------------------

%\begin{rSubsection}{2014 High-Altitude Balloon Project}{December 2013 - May 2014}{Technical Team Lead}{Rapid City, SD}
%\item Led a team of students to design and build the technical payload for a high-altitude balloon. Programmed the flight computer and all the data collection instruments. Constructed the payload module and assembled all technical devices for flight. Analyzed data retrieved. 
%\end{rSubsection}

%---------------------------------------------------------------------------------------

%\begin{rSubsection}{2015 Tornado Shelter Design Project}{December 2014 - June 2015}{Theorist and Component Design}{Rapid City, SD}
%\item Theorist in a team of students for a portable tornado shelter %design competition. Tested and modeled structure design in fluid-flow %software to determine weak componentry. Designed latching and hinge %mechanisms. Calculated fluid dynamic forces on the structure and %determined failure limits for all components. 
%\end{rSubsection}

%----------------------------------------------------------------------------------------

%\begin{rSubsection}{FIRST Robotics Competition}{September 2012 - May 2015}{Programming Team Lead 2014-15}{Rapid City, SD}
%\item {\small Responsible for programming robot in teleoperated and autonomous modes. Worked with other team leads to implement their ideas in order to create functionality and fluidity between different subteams. Assist with robot construction and machining. }
%\end{rSubsection}

\end{rSection}
%----------------------------------------------------------------------------------------






%----------------------------------------------------------------------------------------
%	EDUCATION SECTION
%----------------------------------------------------------------------------------------

\begin{rSection}{Education}

{\bf University of Colorado, Boulder} \hfill {\em August 2015 - Present} \\ 
{\small Engineering Honors Program, Chancellor's Scholarship\\
Masters of Science, Aerospace Engineering Sciences with focus on Fluids and Propulsion \hfill {\textit{Expected Graduation: May 2020}} \\
Bachelor of Science, Aerospace Engineering Sciences \hfill {\textit{Expected Graduation:} May 2019} \\
Bachelor of Music, Violin Performance (College of Music Scholarship Award) \hfill {\textit{Dates:} August 2015 - May 2016} 
%Overall GPA: 3.703
} 

\end{rSection}

%----------------------------------------------------------------------------------------
%	TECHNICAL STRENGTHS SECTION
%----------------------------------------------------------------------------------------

%\begin{rSection}{Technical Strengths}
\begin{rSection}{Strengths and Skills}
\scalebox{0.9}{
\begin{tabular}{ @{} >{\bfseries}l @{\hspace{6ex}} l }
Programming & Python, MATLAB, Bash, C++, LABVIEW, \LaTeX  \\
	Software & ANSYS, OpenFOAM, CAD (SolidWorks), Thermal Desktop, OpenRocket, Mathematica, Microsoft Office Suite \\
Equipment & Microcontrollers, 3D printers, high-vacuum systems, shop equipment, electrical equipment (e.g. soldering) \\
OS & Linux, OSX, Windows, using Virtual Machines \\
Personal Interests & Violin, tuner car modification, freeride skiing, mountain biking, TaeKwonDo, Tai Chi, mountaineering \\

%Computer Languages & Prolog, Haskell, AWK, Erlang, Scheme, ML \\
%Protocols \& APIs & XML, JSON, SOAP, REST \\
%Databases & MySQL, PostgreSQL, Microsoft SQL \\
%Tools & SVN, Vim, Emacs
\end{tabular}
}
\end{rSection}



%----------------------------------------------------------------------------------------
%	EXAMPLE SECTION
%----------------------------------------------------------------------------------------

%\begin{rSection}{Section Name}

%Section content\ldots

%\end{rSection}

%----------------------------------------------------------------------------------------

\end{document}
